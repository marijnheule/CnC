\let\accentvec\vec
\documentclass{llncs}
\usepackage{llncsdoc}
\let\spvec\vec
\let\vec\accentvec

\usepackage{graphicx}
\usepackage{amsmath}
\usepackage{color}
\usepackage{pstricks}
\usepackage{pst-node}
\usepackage{pstricks-add}
\usepackage{algorithm}
\usepackage{algorithmic}
\usepackage{amsfonts}
\usepackage{multirow}
\usepackage{fancyvrb}


\newcommand{\pr}{\mathsf{PR}}
\newcommand{\drat}{\mathsf{DRAT}}

\title{Cube-and-Conquer Tutorial}

\author{Marijn J.H. Heule}

\institute{The University of Texas at Austin}




\begin{document}

\maketitle

\begin{abstract}
This tutorial describes how to obtain, install, and use the cube-and-conquer solver
based on cube solver {\tt march\_cu}. We describe the main options as well as 
some best practices to solve hard problems. The repository also includes the conquer solvers
{\tt iGlucose} and {\tt iLingeling}, scripts that combine the solvers, and benchmarks.
\end{abstract}


\section{Introduction}

Cube-and-conquer is a parallel solving approach which combines the two main complete paradigms to solve 
satisfiability (SAT) problems. The first and most popular paradigm is 
{\em conflict-driven clause learning} (CDCL). Solvers based on the CDCL paradigm can efficiently solve many
problems arising from interesting applications in both industry and science --- including problems that
have millions of variables and clauses. The main strength of CDCL solvers is that they can frequently
find a short refutation cheaply (if one exists) or produce a satisfying assignment. Its core reasoning technique
is generalizing conflicting assignments into conflict clauses, which are added to the formula.
Although effective on a broad range of applications, CDCL solvers tend to show poor performance on
really hard problems, i.e, problems that require more than several hours of computational time. 

The second paradigm, called {\em look-ahead}, focusses on recursively splitting a given formula into 
smaller subproblems. In each step, a splitting variable is selected and the formula is partitioned into
a copy that assigns the splitting variable to true and a copy that assigns the splitting variable to false.
The splitting variables are picked using look-aheads:
A look-ahead consists of the following steps: select a literal; assign it to true; and simplify the
formula under that assignment. The heuristic value of a look-ahead is the difference between the formula before 
and after look-ahead, typically measured by computed a weighted sum of the clauses that are reduced, but not satisfied by
during the simplification.

Cube-and-conquer is more efficient than both pure CDCL and pure look-ahead solvers on many 
hard problems. The name originated from combining the concepts of a cube (a conjunction of literals) and the
divide-and-conquer approach. Given a formula, a look-ahead solver ---also referred to as the cube 
solver--- produces a set of cubes. Each cube is a compact representation of a subproblem:
The subproblem can be obtained by joining the cube and the  formula. The CDCL solver ---also 
referred to as the conquer solver--- solves the subproblems, ideally in parallel.


\section{Installation}

The sources can be obtained by cloning the repository as follows:

\begin{verbatim}
   git clone https://github.com/marijnheule/CnC
\end{verbatim}

\noindent
These sources can be compiled using the build script in the main directory:

\begin{verbatim}
   ./build.sh
\end{verbatim}

\noindent
The same script also allows to remove the binary files (including the executables):

\begin{verbatim}
   ./build.sh clean
\end{verbatim}

\noindent
There is also a script to test whether the tools are working properly. This script
partitions an equivalence checking benchmark {\tt eq.atree.braun.8.unsat.cnf} 
in roughly 0.5 seconds and the conquer solvers can solve the resulting 
subformulas in about 3 seconds.

\begin{verbatim}
   ./test.sh
\end{verbatim}

\section{Basic Usage}
\label{sec:basic}

The CnC repository comes with three tools: one cube (look-ahead) solver 
{\tt march\_cu} and two conquer (incremental CDCL) solvers 
{\tt iGlucose} and {\tt iLingeling}. The basic usage is to split a given 
problem into many subproblems using the cube solver and afterwards
solve the subproblems using one of the conquer solvers. The repository
comes with two scripts to perform these actions, {\tt cube-glucose.sh}
and {\tt cube-lingeling.sh}, which will use {\tt iGlucose} respectively {\tt iLingeling}
to solve the subproblems.

Figure~\ref{fig:EDP} shows an example run on the Erd\H{o}s Discrepancy Problem with 
discrepancy 2 using {\tt cube-lingeling.sh}. This problem was open for 
decades and a Polymath project was started to solve it. The 
discrepancy 2 case was first solved using SAT and cube-and-conquer
is the most efficient way to prove unsatisfiability of transition of SAT
to UNSAT (which is at 1161). Using the default settings, {\tt march\_cu}
generates 64941 cubes in 189.72 seconds. Afterwards,  {\tt iLingeling}
solves the 64941 subproblems in 448.55 seconds using 8 threads
on a MacBook Pro with CPU with 4 cores. The number of threads can
be adjusted by editing the CORES parameter in the {\tt cube-lingeling.sh} script.
The {\tt cube-glucose.sh} script does not support solving cubes in parallel.
However, for most hard problems, this is not an issue as they are solved 
using a two-level partition (see Section~\ref{sec:two-level}).

\begin{figure}[h]
\centering
\begin{verbatim}
./cube-lingeling benchmarks/EDP2_1161.cnf
c down fraction = 0.020 and down exponent = 0.300
c cubes are emitted to /tmp/cubes42340
c CNF of 4207 variables and 25142 clauses based on DIMACS p-line.
c equivalence reasoning is turned off.
c magic constants: min = 8.00, bin = 25.00, and max = 550.00
c init_lookahead: longest clause has size 3
c number of free variables = 4207
c highest active variable  = 4207
c dynamic_preselect_setsize :: off
c |****************************************************************.|
c
c main():: nodeCount: 64941
c main():: dead ends in main: 36
c main():: lookAheadCount: 228251450
c main():: unitResolveCount: 3913316
c time = 189.72 seconds
c main():: necessary_assignments: 57277
c print learnt clauses and cubes
c number of cubes 64941, including 1519 refuted leaves
c average weight cubes 4015.178, average weights leaves 3977.193
c 64941 / 64941 |====================| 100% 0.0550 sec/cube 
s UNSATISFIABLE

real	7m28.545s
user	56m6.822s
sys	0m19.779s
\end{verbatim}
\caption{Example run of {\tt cube-lingeling.sh} on an Erd\H{o}s discrepancy formula}
\label{fig:EDP}
\end{figure}

Be aware that look-ahead heuristics are very expensive. As a consequence it pays off 
to simplify the formula before splitting it into many subproblems. For 
really large problems this is absolutely crucial. The main folder contains
a script that uses {\tt lingeling} to simplify the formula using various 
variable and clause elimination techniques and short (incomplete) runs of CDCL. 
The default use of the simplification script is as follows:

\begin{verbatim}
   ./simplify.sh <input-file>
\end{verbatim}

This script produces the file {\tt reduced.cnf} in the main directory.

\section{Motivating Example and Hidden Strength}

Cube-and-conquer is not only useful to partition a hard problem into many subproblems that can be solved in parallel, 
but also to improve performance of solving a (sub)problem on a single core. Let $N$ be the number of cubes in a partition. 
The problem is solved with pure CDCL if $N=1$, while it is solved
using pure look-ahead for very large $N$.

An interesting pattern can be observed when increasing $N$, while running on a single core: First the total runtime goes up. 
Apparently some subproblems are as hard as the original one. Afterwards, if $N$ become large enough, then the total runtime goes down and could become 
significantly smaller compared to solving it with CDCL alone (again running both on a single core). At some point, when
$N$ becomes really large, then runtime goes up again. At this point the cost of splitting starts to kick in and dominates 
the total costs.
Figure~\ref{fig:hidden} shows this pattern on a Schur number problem. Schur number problems ask whether
the positive integers can be colored with a finite number of colors without resulting in a monochromatic solution of the equation
$a  + b = c$.

The times in Figure~\ref{fig:hidden} (y-axis) refer to the total runtime of solving all subproblems sequentially on a single CPU.
Solving this problem with pure CDCL ($N=1$) requires roughly 1000 seconds. Splitting the problem into ten subproblems ($N=10$)
and solving them increase the total computational costs to about 2500 seconds because 3 subproblems were almost as hard as the 
original problem. This demonstrates that cube-and-conquer can be harmful when generating only a few subproblems.
The total runtime starts to decrease by further increasing the number of subproblems. For this particular problem, the optimal 
number of cubes is close to 10,000, which allows solving the problem in about 150 seconds on a single core. By solving these 
subproblems in parallel, one can solve this problem in a few seconds. 

A potential explanation of this plot is as follows. In case of few subproblems ($N=1$ and $N=10$) there exist no short refutations.
As a consequence CDCL solvers struggle to solve (some of) these subproblems. Further splitting the problem will introduce short
refutations that can be quickly found using CDCL. Once there are very short refutations, then splitting makes little sense and will 
only increase the splitting costs. 

\begin{figure}[h]
\centering
\includegraphics[width=.7\columnwidth]{histogram}
\vspace{-10pt}
\caption{Comparison of the total runtime in seconds (y-axis) of solving a Schur number problem using a different number of cubes (x-axis) on a single core (no parallelism).}
\label{fig:hidden}
\end{figure}

\section{Tool Chain and File Formats}

The cube-and-conquer tool chain consists of three phases: simplification, cubing, and solving. The simplification phase is optional. 
Input files for the simplification and the cubing phases should be in the {\sc dimacs} format. Figure~\ref{figure:input-formats} shows
an example input file ({\tt tests/Rivest.cnf}). The syntax of {\sc dimacs} is shown below. 

\begin{figure}[!htb]
\centering
~~~
\begin{minipage}[t]{.22\textwidth}
\centering
\vspace{-67pt}
{ {\sc dimacs}}
\begin{Verbatim}[frame=single]
 p cnf 4 8
  1  2 -3 0
 -1 -2  3 0
  2  3 -4 0
 -2 -3  4 0
 -1 -3 -4 0
  1  3  4 0
 -1  2  4 0
  1 -2 -4 0
  
\end{Verbatim}
\end{minipage}
~~
\begin{minipage}[t]{.22\textwidth}
\centering
\vspace{-67pt}
{ {\sc inccnf}}
\vspace{-0.048cm}
\begin{Verbatim}[frame=single]
 p inccnf
  1  2 -3 0
 -1 -2  3 0
  2  3 -4 0
 -2 -3  4 0
 -1 -3 -4 0
  1  3  4 0
 -1  2  4 0
  1 -2 -4 0
 a 1 0
 \end{Verbatim}
\end{minipage}
\hfill
\begin{minipage}{.45\textwidth}
\caption{A propositional formula in the  {\sc dimacs} format (left) and the  {\sc inccnf} format (right). 
These formats differ in the header and the {\sc inccnf} format facilitates to including cube lines (starting with {\tt a}).
Positive numbers refer to positive literals,  while negative literals refer to negative literals.
Each non-header line terminates with a {\tt 0}. }
\label{figure:input-formats}
\end{minipage}
%\vspace{-20pt}
\end{figure}


\begin{eqnarray*}   
   \langle \mathrm{dimacs} \rangle           &=& ``\mathrm{p~cnf}" \{\langle \mathrm{pos} \rangle\}  \{\langle \mathrm{pos} \rangle\} \{ \langle \mathrm{clause} \rangle \}\\
   \langle \mathrm{clause} \rangle         &=&  \{  \langle \mathrm{lit} \rangle \} ``0"\\
   \langle \mathrm{lit} \rangle                 &=&  \langle \mathrm{pos} \rangle \mid  \langle \mathrm{neg} \rangle\\
   \langle \mathrm{pos} \rangle              &=&  ``1" \mid  ``2" \mid \dots \mid  \langle \mathrm{max-idx} \rangle\\
   \langle \mathrm{neg} \rangle              &=& ``\mathrm{-}" \langle \mathrm{pos} \rangle\\
\end{eqnarray*}


\noindent
The interface between the cube and conquer solvers are {\sc cube} files, which consist of a
sequence of cubes. Files in the {\sc cube} format have the following syntax.
\begin{eqnarray*}   
   \langle \mathrm{cube} \rangle         &=& ``\mathrm{a}" \{  \langle \mathrm{lit} \rangle \} ``0"\\
\end{eqnarray*}
The ``a" in {\sc cube} files refers to {\em assumption}. The intended meaning is that the 
incremental SAT solver will solve the formula under the assumption that all literals in the
cube are true. If the solver cannot find a solution under these assumptions, then at least
one of the literals must be assigned to false in any solution of the formula. 

The incremental versions of {\tt Glucose} and {\tt Lingeling},  named
 {\tt iGlucose} respectively {\tt iLingeling}, require input files in the {\sc inccnf}
format. The {\sc inccnf} format modifies the the {\sc dimacs} format in two 
ways. First, the header line (the line starting with {\tt p cnf}) is replaced the
line {\tt p inccnf}. In contrast to {\sc dimacs}, the header line in {\sc inccnf} files
does not contain the number of variables and clauses. Second, the {\sc inccnf} files
can include cubes. The syntax is shown below.
\begin{eqnarray*}   
   \langle \mathrm{inccnf} \rangle         &=& ``\mathrm{p~inccnf}" \{  \langle \mathrm{clause} \mid  \mathrm{cube} \rangle \}\\
\end{eqnarray*}

Although the {\sc inccnf} format allows interleaving clauses and cubes, this feature is not used here. 
In the cube-and-conquer setting, the {\sc inccnf} consist of two blocks of which the first one
is a sequence of clauses and the second one a sequence of cubes. 

\section{Splitting Options}

The default options are set to result in an effective splitting on a wide
spectrum of formulas. However, changing some of the heuristics can
generally improve the quality of the splitting for specific instances. This
section provides on overview of the options, while the next section 
offers some specific guidance how to modify the heuristics. 

The command line options of {\tt march\_cu} are listed by using the {\tt -h} option
and are shown below. 


\begin{verbatim}
c march_cu help
c USAGE: ./march_cu <input-file> [options]

   where input may be in (uncompressed) DIMACS.

c OPTIONS:

   -h            prints this help message
   -p            plain / no cube mode
   -d <int>      set a static cutoff depth (default:    0, dynamic depth)
   -n <int>      set a static cutoff vars  (default:    0, dynamic depth)
   -e <float>    set a down exponent       (default: 0.30,   fast cubing)
   -f <float>    set a down fraction       (default: 0.02,   fast cubing)
   -l <int>      limit the number of cubes (default:    0,      no limit)
   -s <int>      seed for heuristics       (default:    0,     no random)
   -#            #SAT preprocessing only

c OPTIONAL LOOKAHEAD TECHNIQUES (option will negate the default):

   -gah          global autarky heuristic  (default: on)
   -imp          add both implications     (default: on)
   -wfr          add windfall resolvents   (default: on)

c OUTPUT OPTIONS:

   -o <file>     emit the cubes to <file>  (default: /tmp/cubes.icnf)
   -q            turn on quiet mode        (set default output to stdout)
   -cnf          add the cnf to the cubes

c MAGIC CONSTANTS:

   -bin <float>  binary clause weight      (default:  25.00)
   -dec <float>  size exponential decay    (default:   0.50)
   -min <float>  minimum heuristic value   (default:   8.00)
   -max <float>  maximum heuristic value   (default: 550.00)
   -sli <int>    singlelook iterations     (default:      9)
   -dli <int>    doublelook iterations     (default:      2)
\end{verbatim}

\subsection{Main Options}
\label{sec:opt:main}

A crucial heuristic in cube-and-conquer is determining when to cut off splitting and solve the resulting subformula using CDCL. 
The heuristic is dynamic and update rely on a technique called failed literal elimination. Recall that a look-ahead consists of
picking a literal $l$; assigning it to true; and simplify the formula under that assignment. Whenever the simplification results in a
conflict, then literal $l$ is called a failed literal. Look-ahead solvers force failed literals to false and simplify the formula, known as
failed literal elimination. This simplification can also result in a conflict. In that case we state that failed literal elimination refutes
that node (or the subformula represented by that node).

The dynamic heuristic assumes the following: if failed literal elimination is able to refute a certain subformula, 
then CDCL can solve that subformula as well as similar subformulas more efficiently. Each time failed literal elimination refutes
a subformula a heuristic value of the formula is computed. This heuristic value can be based on various statistics, such as the
number of variables or the number of binary clauses. A favorable property of the heuristic value is that it only decreases when 
descending in the search-tree. This holds for the number of variables,  but not for the number of binary clauses. In this subsection
we assume that the heuristic value is the measured by counting the number of variables (which is the default in {\tt march\_cu}).

\subsubsection{Limiting the cube generation.}

Generating subproblems is the main functionality of {\tt march\_cu}. However, the tool can be turned into a pure look-ahead
solver using the {\tt -p} option (short for plain) that turns off cubing and thus splits every node until failed literal elimination can refute the node.

Two options in {\tt march\_cu} will replace the dynamic heuristic by a static one. The first such option is {\tt -d D}, which will 
cut off splitting after depth {\tt D}. For example, running the solver using option {\tt -d 10} produces at most $1024~(= 2^{10})$
cubes. The second such option is {\tt -n N}, which will cut off splitting after the number of variables drops below {\tt N}.
For both these options holds the following: If failed literal elimination is able to refute a subformula, then that subformula is not
split resulting in fewer cubes in the output. As a consequence, some cubes may have fewer than {\tt D} literals when using the
{\tt -d D} option and subformulas may have more than {\tt N} variables when using the  {\tt -n N} option.

% Mention that N is the number of variables as in the current state of march_cu

The option {\tt -l L} post-processes the resulting cubes by emitting at most {\tt L} cubes. This option only kicks in after the cubing
is finished and when the number of generated cubes is larger than {\tt L}. For each node (including the internal nodes) in the
search-tree the heuristic value is stored. The {\tt -l L} option repeats the following: while the number of cubes is larger than {\tt L};
let $v$ be an internal node with the highest heuristic value; remove both leaf nodes that are children of $v$ from the search-tree.


\subsubsection{The update mechanism.}
The dynamic heuristic is based on a mechanism that approximates the optimal number to realize fast performance.
The mechanism stops splitting if the number of remaining variables in a node (subproblem) drops below the value of parameter $\delta$. 
%The number of remaining variables is a useful measure as it strictly decreases, whereas
%number of binary clauses can oscillate. Initial experiments showed that such oscillation can slowdown the solver on easy problems.
We initialize $\delta$ to $0$, meaning that we keep splitting until $\delta$ is increased. We only increase $\delta$ when failed literal elimination
can refute a node, which naturally terminates splitting. In this case, $\delta$ is set to the number of remaining variables in that node.
%The increase is motivated as follows: If look-ahead techniques can refute a node, then we expect CDCL to
%refute that node---as well as similar nodes---more efficiently.
%
The value of $\delta$ is decreased in each node of the search-tree to ensure that  failed literal elimination
refutes nodes once in a while. Experiments with various 
methods to implement the decrement showed  that the size of the decrease should 
be related to the depth of a node in the search-tree. The closer a node is to the root, the larger the decrement.
The following update function is used in {\tt march\_cu} (with $d$ referring to the depth of the node that performs the update and 
parameter ${\tt E}$ referring to  {\em down exponent} and parameter ${\tt F}$ referring to the {\em down factor}): 
%
\[
\delta := \delta \big(1 - {\tt F}^{d^{\tt E}}\big)
\]
%
If the value of ${\tt F}$ is close to $0$, then $\delta$ climbs to a value at which look-ahead techniques 
will rarely refute a node. On the other hand, if the value of ${\tt F}$ is close to $1$, then $\delta$ drops quickly 
to $0$, so that practically all leaf nodes will be refuted by look-ahead techniques. The value of ${\tt E}$ determines
the influence of the depth. If ${\tt E}$ is close to $0$, then the depth is ignored during the update, while if ${\tt E}$ is
close to $1$ (or even larger), then the depth is dominant.  
The default values for the down exponent and down factor in {\tt march\_cu} are: ${\tt E} = 0.3$ and ${\tt F} = 0. 02$.
These values result in fast cubing. Examples of combinations of values that result in similar partitions are:
${\tt E} = 0.5$ and ${\tt F} = 0.1$ as well as ${\tt E} = 1.0$ and ${\tt F} = 0.6$. To change the down exponent and down factor
use the options {\tt -e E} and {\tt -f F}, respectively.


\subsubsection{Random value selection heuristic.} Given a decision variable, {\tt march\_cu} assigns it to the truth value that results
in the largest reduction. This generally produces short cubes first and long cubes near the end. In some cases, this might
result in a far from linear progress bar. This can be countered by making the truth value assignment random. Providing a seed
using the {\tt -s S} option turns on random truth values with seed {\tt S}.

\subsubsection{Splitting and counting.} Some techniques in {\tt march\_cu} eliminate solutions, although they preserve satisfiability.
An example of such a technique is pure literal elimination: If a variable occurs only in one polarity, then it is fixed to that polarity. 
These techniques cannot be used, when one wants to count the number of solutions of a formula using cube-and-conquer. The
option {\tt -\#} turns off all techniques that elimination solutions.


\subsection{Optional look-ahead techniques}

The heuristics in the {\tt march\_cu} solver have been optimized over the years to achieve fast performance on a wide spectrum
of formulas --- as a standalone solver. Effective heuristics for solving are generally also effective heuristics for splitting. However,
we observed that this may not be the case for the three heuristics described in this subsection. The options allow to turn
them off. 

An autarky is an assignment that satisfies all clauses that are touched (i.e., have at least one assigned literal) by the assignment.
In case an autarky is found, then one can remove all clauses touched by the autarky, while preserving satisfiability. Autarkies are
potentially very powerful, but hardly observed in practice. However, the technique {\em global autarky heuristic} that tries to increase
the probability of finding an autarky is generally quite effective. The global autarky heuristic restricts decision literals to occur in 
clauses that are touched, but not satisfied by the current assignment. A side-effect of this heuristic is that it can significantly reduce
the look-ahead costs on large formulas as the restriction is also applied on look-aheads. The global autarky heuristic amplifies the 
effect of the decision heuristics as future decisions will depend on earlier decisions. This also means that the global autarky heuristic
tend to make poor heuristics worse. The option {\tt -gah} allows turning this heuristic off. 

Look-aheads are not only used to determine decision variables, but also to learn local binary clauses, also known as windfall resolvents.
These local binary clauses help solve and split problems faster. However, turning off learning local binary clauses can be helpful to generate easier subproblems.
The option {\tt -wfr} fully turns off adding windfall resolvents, while the option {\tt -imp} turns off the addition of the less useful windfall resolvents.

\subsection{Output options}

When {\tt march\_cu} is running in the cube producing mode (the default), then it emits a sequence of cubes to the default 
output file {\tt /tmp/cubes.icnf}. The option {\tt -o} allows to change the output file. Alternatively, the cubes can be printed to
stdout using the {\tt -q} option. This will suppress all other output to stdout. The output file does not include the {\tt p inccnf}
header and the clauses of the input formula. Both of them can be added to the output file using the {\tt -cnf} option.

\subsection{Magic constants}
\label{sec:opt:magic}

The look-ahead evaluation function measures the difference between a formula before and after a look-ahead. 
The difference is computed by the weighted sum of {\em new} clauses, i.e, those clauses that exist in the formula after the look-ahead, but
not in the formula before the look-ahead. In other words, new clauses are the clauses that were reduced in length, but not satisfied
during a look-ahead.

The difference is computed by multiplying the weights of literals in new clauses. These literal weights are computed in each node
of the search tree. Occurrences in smaller clauses result in higher weights. The weight of occurring in a ternary clause is fixed to 1.
The {\tt -bin} option allows setting the weight of occurring in a binary clause. The observed best weight for binary clauses is 3.3 for random $k$-SAT formulas.
%This default value appears to be useful for many non-random formulas as well.
However, for some other formulas higher weights will improve the quality of the splitting. For example, using weight 8 instead of 3.3
significantly improves performance on the Boolean Pythagorean Triples problem. The default value is 8.

In case a literal occurs in long clauses, i.e., clauses with length $l > 3$, then the weight is set to $w^{l-3}$ in which $w$ is the exponential decay parameter 
that can be set using the {\tt -dec} option. The default value for {\tt -dec} is 0.5, but lower values have been observed to improve the 
performance on some problems.

The literal weights are computed using an iterative process. At the end of each iteration the weights are normalized. After normalization
some weights tend to be very small or very large, which reduces the effectiveness of the heuristics. Two magic constants are introduced
to make sure that literal weights are in between a certain range after normalization. The minimal value of this range can be set using
the {\tt -min} option, while the maximum value of this range can be set using the {\tt -max} option. The default values tend to work
well on many problems. 

The final magic constants influence the accuracy of look-ahead technique. Eliminating a failed literal makes the heuristic values
of look-aheads inaccurate as they have been computed on a different formula. It typically pays off to recompute the heuristic 
values, because more accurate heuristic values result in better decisions. Also, a failed literal may turn other literals into failed
literals too, which can be computed by recomputing the heuristic value. However, the number of times the 
heuristic value can be recomputed in a single node of the search-tree should be limited in order to avoid extreme cases.
This limit is by default 9 for single depth look-aheads  and 2 for double depth look-aheads. These values can be changed using
the command line options {\tt -sli} and {\tt -dli}, respectively. For most heuristics hold that: what is good for pure look-ahead is 
good for cubing. However, this is not the case for these two limits. For pure look-ahead these limits should be high, while for 
cubing these values should be low. For several formulas it is best to set them to 0, thereby turning the recomputation of 
the heuristic values off.

\section{Solving Hard Problems using Cube-and-Conquer}

Cube-and-conquer has been developed to solve hard problems efficiently and to achieve linear-time speedups even when using
thousands of cores. Other paradigms, such as CDCL, are likely to be more effective on easy problems. This section offers some 
options to improve the performance to cube phase of cube-and-conquer for hard problems. 

\subsection{Determine the Hardness}

The first step of solving a hard problem is to estimate its hardness. A rough estimate of the hardness is the fraction of nodes that can
be refuted using failed literal elimination. For example, consider the following line in
the output of {\tt march\_cu} (also in Figure~\ref{fig:EDP}):
%
\begin{verbatim}
   c number of cubes 64941, including 1519 refuted leaves
\end{verbatim}
%
In this case the number of nodes that were refuted using failed literal elimination is 1519 (out of 64941 nodes in the search-tree). 
These numbers suggest that the problem is easy. 

For hard problems it will be costly to run {\tt march\_cu} using the default parameter settings. In order to determine the hardness 
of such problems, it is best to limit the cube generation, for example using the {\tt -d} option (see Section~\ref{sec:opt:main}).
Below are the statistics of running {\tt march\_cu} with various depth limits on the Boolean Pythagorean Triples problem, which requires roughly 4 CPU
years to solve (using cube-and-conquer and optimized heuristics). 
%
\begin{verbatim}
  -d 16:  c number of cubes 65516, including 8 refuted leaves
  -d 17:  c number of cubes 130997, including 32 refuted leaves
  -d 18:  c number of cubes 261883, including 123 refuted leaves
  -d 19:  c number of cubes 523030, including 768 refuted leaves
  -d 20:  c number of cubes 1041915, including 4016 refuted leaves
\end{verbatim}
%
Alternatively, one can limit the cube generation based on the number of variable using the {\tt -n} option (see Section~\ref{sec:opt:main}).

Problems for which {\tt march\_cu} cannot refute any subproblem at a reasonable depth (say 20) should be considered too hard
and not solvable with existing techniques. 

\subsection{Optimizing the Heuristics}

The default heuristics in {\tt march\_cu} aim to produce effective splitting on a wide range of propositional formulas. However, for 
most problems it pays off to manually optimize the heuristics to improve the quality of the splittings. The most important 
heuristic parameters can be updated via the command line using the options described in Section~\ref{sec:opt:magic}.
Optimizing the heuristics requires selecting useful subproblems of the hard formula.
This can be done as follows: Fist determine the depth for which the number of refuted nodes is at least 1000. In the above example
of the Boolean Pythagorean Triples problem, this depth is 20. Second, randomly pick about 100 subproblems (cubes) of the partition on that depth.
Second, solve these 100 subproblems and select the 10 hardest ones for the optimization.

The repository contains a script that produces the corresponding subproblem of a given formula and cube by simply adding
the literals in the cube as unit clauses to the formula. An example execution of the script is shown below.

\begin{verbatim}
./apply.sh tests/Rivest.cnf tests/Rivest.cubes 3 > tests/sub3.cnf
\end{verbatim}

It is important to consider both the cube solver and conquer solver times when optimizing the heuristics. 
The {\tt cube-glucose.sh} script is useful for this purpose. This script will forward all the command line
options to the cube solver {\tt march\_cu}. For example, the command 
%
\begin{verbatim}
./cube-glucose.sh tests/ptn-easy.cnf -dli 0
\end{verbatim}
%
will increase the runtime of the cube solver, but reduces the runtime of the conquer solver. The total
runtime, however, is similar compared to running the script with the default setting of {\tt -dli 2}.

Optimizing the heuristics fully automatically is a direction of future research.

%The {\tt march\_cu} solver can be used for this purpose by
%running it with a very small down factor (the {\tt -f F} option, see Section~\ref{sec:opt:main}).
%The update mechanism assumes that failed literal elimination can regularly refute nodes of the search-tree. If that is not the case
%then cubing will be very costly. There are generally two reasons for which the assumption fails: the problem is too hard or the 
%heuristics are poor. This section offers some suggestions to deal with the second reason. 

\subsection{Two-Level Partition}
\label{sec:two-level}

In order to achieve strong performance, it is important to balance the costs of the cube and conquer phases. For many problems
it appears that a 50\%/50\% balance is effective. As cube generation will consume about half the computational resources, it should be performed in parallel (not only solving
the subproblems). This can be achieved using a two-level partition: 1) compute a top-level partition by limiting the cube 
generation using either the {\tt -d} or {\tt -n} option (on a single core); and 2) partition each of the resulting subproblems in parallel without
limiting the cube generation. In general it is better to use the {\tt -n} option for creating the top-level partition as the subproblem
that are generated using the {\tt -d} option tend to differ a lot in difficulty. Determining a useful limit for the {\tt -n} option can be somewhat challenging. 
The easiest option is to use the ``average weight cubes" reported by {\tt march\_cu} (see Figure~\ref{fig:EDP}), while generating cubes
using the {\tt -d} option.

\section{Proofs and Validation}

The {\tt iGlucose} solver supports emitting a proof of unsatisfiability by extending the
solver call with {\tt -certified -certified-output=proof | grep -v bound}. This produces the 
proof of unsatisfiability in the file {\tt proof} and ignores all output of solving the subformulas. 
The repository includes a script to produce a proof automatically. For example:

\begin{verbatim}
./cube-glucose-proof.sh tests/eq.atree.braun.8.unsat.cnf 
\end{verbatim}

The {\tt proof} file is in the {\sc drat} format and can be checked with the tool {\tt drat-trim}. 
This tool is available at \url{https://github.com/marijnheule/drat-trim}. After installing this
tool in the home directory, the proof can be checked using the following command:

\begin{verbatim}
~/drat-trim/drat-trim tests/eq.atree.braun.8.unsat.cnf proof
\end{verbatim}

This results in the following proof checking log:

\begin{verbatim}
c parsing input formula with 684 variables and 2300 clauses
c finished parsing
c detected empty clause; start verification via backward checking
c 2248 of 2300 clauses in core                            
c 118143 of 133661 lemmas in core using 13573339 resolution steps
c 0 RAT lemmas in core; 117929 redundant literals in core lemmas
s VERIFIED
c verification time: 7.140 seconds
\end{verbatim}

\section{Additional Tools}

Cube files can become very large when solving hard formulas. The repository contains lightweight compression tools to
significantly shrink these files. The tools are based on two compression techniques. First, the cube files generated by 
{\tt march\_cu} form an assignment tree and each of the cubes represent a leaf node in this assignment tree. Storing the 
tree instead, using only one literal per node of the tree results in a substantial reduction. Second, instead of storing
the numbers in ASCII format, they can be stored a binary format. This optimization results to an additional
reduction of $40\%$ on average.

A cube file can be compressed using the following command:
%
\begin{verbatim}
   ./tools/cube-encode test.cubes test.binc
\end{verbatim}
%
where the input file {\tt test.cubes} is  a sequence of cubes that form an assignment tree (as emitted by {\tt march\_cu}). The output
file {\tt test.binc} is a binary file.

Decompressing {\tt  test.binc} into {\tt test.cubes} can be performed using:

\begin{verbatim}
   ./tools/cube-decode  test.binc > test.cubes
\end{verbatim}

%
%\section{Conclusions}

\bibliography{references}
\bibliographystyle{splncs}

\end{document}
